%D�finir le format du document: papier, taille de police, type de document, etc.
\documentclass[a4paper, 11pt]{article}

%%%%%%%%% Packages externes utilis�s %%%%%%%%%%%%%%%%%%%
\usepackage[french]{babel}
\usepackage[latin1]{inputenc}
\usepackage[T1]{fontenc}
\usepackage{verbatim}
\usepackage{graphicx}
\usepackage{epstopdf}
\usepackage{macro}
\usepackage{algorithm}
\usepackage{algorithmic}
%\usepackage{algorithm2e}

%La mise en page du rapport, NE PAS MODIFIER.
\usepackage{geometry}
 \geometry{
 a4paper,
 left=20mm,
 right=20mm,
 top=20mm,
 bottom=20mm,
 }

%%%%%%%%% Le corps du document entre begin et end %%%%%%%%%%%%%%%%%%%
\begin{document}

%Page de garde
%%%%%%%%%%%%%%% Page de garde %%%%%%%%%%%%%%%%%%%

\begin{titlepage}{
    \begin{center}
        \vspace* {25mm}
        {\Large \textbf {Universit� de Cergy-Pontoise}} \\
        \vspace* {10mm}
        {\Large \textbf {RAPPORT}} \\
        \vspace* {10mm}
        pour le projet G�nie Logiciel \\
        \textbf {Licence d'Informatique deuxi�me ann�e} \\
        \vspace* {10mm}

	sur le sujet \\
        \vspace* {10mm}
	{\Huge \textsf{Messagerie}} \\
        \vspace* {10mm}
 	r�dig� par\\
        \vspace* {10mm}
        {\Large \textbf {Louis L'Haridon, Julien Demoineret et Valentin Petiteville} \\
        \vspace* {5mm}
        tuteur� par\\
        \vspace* {5mm}
        {\Large \textbf {Tianxiao Liu} \\
				\vspace* {10mm}
				\noreffig{images/logo.png}{12.82cm}{8.2cm} \\
        \date Mai 2017
        \vspace* {10mm}
	\end{center}
}
\end{titlepage}


%G�n�ration automatique de la table des mati�res, de la liste des figures et de la liste des tableaux
\tableofcontents
\listoffigures
\listoftables

%Une section "remerciements" pourrait �tre int�ressante. C'est une section non num�rot� (avec un * )
\section*{Remerciements}
Nous voudirons remercier Monsieur Tianxiao LIU pour nous avoir accompagn� tout au long de ce projet et pour avoir su nous aider lors de difficult�s.

\section{Introduction}
\label{sec:introduction}

%Il n'y a pas d'espace au d�but du paragraphe.
\noindent Ceci est une d�monstration d'un document r�dig� en utilisant LaTex.

Pour commencer un nouveau paragraphe, il suffit de sauter une ligne.

\paragraph{Fonctionnalit�s} Fonctionnalit�s du programme:
\begin{itemize}
\item Capable de disposer des objets dans un environnement simul�.
\item Syst�me de messagerie pour permettre une conversation entre l'assistant de la maison et l'utilisateur.
\item Automatisation de t�ches issues de la routine de l'utilisateur.
\item affichage de l'�tat d'un objet (allum�, �teint ou probl�me).
\end{itemize}

\paragraph{} Dans \cite{data}, les auteurs ont voulu montrer blabla...

\paragraph{} Dans la formule math�mtique \ref{eq:f1}, on peut voir que blabla...

\begin{equation}
\cos (2\theta) = \cos^2 \theta - \sin^2 \theta
\label{eq:f1}
\end{equation}

\paragraph{} Outils de d�veloppement:
\begin{enumerate}
\item Java
\item Eclipse
\item Latex
\end{enumerate}

\section{Sp�cification}
\label{sec:specification}

\paragraph{Chapeau} Nous avons pr�sent� l'objectif du projet dans la section \ref{sec:introduction}. Dans cette section, nous pr�sentons la sp�cification de notre logiciel r�alis�. Ceci correspond principalement au cahier des charges.

\subsection{Premi�re sous-section}
\label{sec:spec1}

\paragraph{Premier paragraphe} On commence � expliquer...

\paragraph{} Juste un simple paragraphe.

\subsection{Deuxi�me sous-section}
\label{sec:spec2}

\begin{table}[h!]
\centering
\begin{tabular} {|p{3.5cm}|p{2.5cm}|p{5cm}|}
\hline
Document & Coefficient & Commentaire \\
\hline
Cahier des charges & 37.5\% & Premier document \\
\hline
Rapport & 62.5\% & Rapport final du projet \\
\hline
\end{tabular}
\caption{Documents � remettre}
\label{tab:document}
\end{table}

Comme ce qui est illustr� dans le tableau \ref{tab:document}, ...

\section{R�alisation}
\label{sec:impl}

%\begin{figure}
%\centering
%\includegraphics[width=3.5cm, height=2cm]{images/programmer.png}
%\caption{Un programmeur occup�}
%\label{fig:modele}
%\end{figure}




\paragraph{} Dans la figure \ref{fig:programmeur}, on peut voir un programmeur tr�s occup� par son travail.


%%% Une autre fa�on pour �crire un algorithme %%%
%\begin{algorithm}[H]
 %\KwData{this text}
 %\KwResult{how to write algorithm with \LaTeX2e }
 %initialization\;
 %\While{not at end of this document}{
  %read current\;
  %\eIf{understand}{
   %go to next section\;
   %current section becomes this one\;
   %}{
   %go back to the beginning of current section\;
  %}
 %}
 %\caption{How to write algorithms}
%\end{algorithm}
\section{Manuel Utilisateur}
\label{sec:manuel}

\noindent Cette section est d�di�e au manuel utilisateur. 

\section{D�roulement du projet}
\label{sec:deroulement}

\noindent Dans cette section, nous d�crivons comment la r�alisation du projet s'est d�roul�e au sein de l'�quipe de projet. La r�partition des t�ches, la synchronisation du travail et l'utilisation du temps seront abord�es. 

\section{Conclusion}
\label{sec:conclusion}

\noindent Dans cette section, nous r�sumons la r�alisation du projet et nous pr�sentons �galement les extensions et am�liorations possibles du projet.



%R�f�rences bibliographiques du document
\bibliographystyle{plain}
\bibliography{bibliographies}
\nocite{*}

\section{Annexe}
\label{sec:annexe}
\paragraph{}Cahier des charges � venir...

\end{document}
