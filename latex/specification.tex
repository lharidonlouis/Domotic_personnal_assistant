\section{Interface Graphique}
\label{sec:specification}

\paragraph{} Pour la partie graphique, la fen�tre principale est compos�e de deux partie (cf photo).

\subsection{Partie fixe}
\label{sec:spec1}

\paragraph{}La partie de gauche, qui est fixe, constitue la cartographie de la maison. Elle est facilement modifiable afin de s'adapter � tout autre disposition de l'utilisateur. Les pi�ces, implant� ainsi permettent une visualisation ais�e des diff�rentes parties de cette environnement, la disposition ainsi que le choix des couleurs s'est fait en fonction du gout du groupe. 
\paragraph{D�tail Plan}Cette habitation est donc compos�e d'un grand s�jour, une salle de bain, une chambre et un grand garage. Les carr�es noirs sur les ext�rieurs repr�sentent les contours de la maison tandis que ceux pr�sent dans cet ensemble repr�sente les murs de la maison. Les carr�es vides au sein des murs sont les portes. A terme des carr�es lumineux indiqueront l'�tat de l'objet dans la pi�ce. Il y aura trois �tats au total, un rouge, un orange et un bleu. 

\subsection{Partie flexible}
\label{sec:spec2}

\renewcommand{\labelitemi}{\textbullet}
\begin{itemize}
\item \bf L'assistant
\end{itemize}

\paragraph{}La seconde partie de l'interface est plus flexible puisqu'elle est compos�e de diff�rents panels. Le panel de d�part comprend une interaction avec deux boutons et un formulaire de discussion avec l'assistant. Celui-ci permet de discuter, il analyse notre phrase pour nous proposer une phrase correspondante. Son vocabulaire est facilement modifiable ainsi que des phrases pr�d�finies. L'assistant est capable d'interagir avec nous sur des sujets basiques et serra capable d'approfondir ses r�ponses par le biais de support externe, mais nous y reviendrons plus tard. 

\newpage
\begin{itemize}
\item \bf Les boutons
\end{itemize}

\paragraph{Description}En ce qui concerne les boutons, le premier permet d'afficher toutes les actions des objets pr�sent dans la maison, dans un nouveau panel. Ainsi on pourra tr�s facilement, allumer ou �teindre diff�rents objets et m�me mettre des sc�nes directement pour avoir un effet imm�diat comme par exemple une sc�ne pour manger ou pour regarder la t�l� qui changera l'intensit� des lumi�res ou activeras ou non certains objets, ces sc�nes se feront au bon vouloir de l'utilisateur, l� encore le choix est libre. 
\paragraph{2�me bouton}Le second bouton permet d'ajouter des objets, cette fonctionnalit� est suppl�mentaire, elle fait partie des extensions que nous envisageons d'int�grer plus tard � notre programme, afin de le rendre encore plus r�elle, fonctionnelle et parfaitement utilisable au sein d'une habitation en constante �volution, avec des besoins qui change tous les jours. 


\fig{images/Home.png}{15cm}{10cm}{Interface du logiciel Home}{Home}
